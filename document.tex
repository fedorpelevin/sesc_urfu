\documentclass[a4paper,twocolumn,12pt]{article}			% класс документа и размеры страницы и шрифта
\pagestyle{headings}							% оформление страницы, включая колонтитулы и нумеровку страниц


\usepackage{float}
\usepackage{multirow}

\usepackage[left=2cm,right=2cm,
    top=2cm,bottom=2cm]{geometry}
    
\usepackage{csquotes} % ещё одна штука для цитат
    

%%% Работа с русским языком
\usepackage{cmap}                                       % поиск в PDF
\usepackage{mathtext}                           % русские буквы в фомулах
\usepackage[T2A]{fontenc}                       % кодировка
\usepackage[utf8]{inputenc}                     % кодировка исходного текста
\usepackage[english,russian]{babel}     % локализация и переносы

%%% Дополнительная работа с математикой
\usepackage{amsfonts,amssymb,amsthm,mathtools} % AMS
\usepackage{amsmath}
\usepackage{icomma} % "Умная" запятая: $0,2$ --- число, $0, 2$ --- перечисление


%% Свои команды
\DeclareMathOperator{\sgn}{\mathop{sgn}}
\newtheorem{definition}{Определение} % задаём выводимое слово (для определений)


%%% Работа с картинками
\usepackage{graphicx}  % Для вставки рисунков
\graphicspath{{images/}{images2/}}  % папки с картинками
\setlength\fboxsep{3pt} % Отступ рамки \fbox{} от рисунка
\setlength\fboxrule{1pt} % Толщина линий рамки \fbox{}
\usepackage{wrapfig} % Обтекание рисунков и таблиц текстом

%%% Работа с таблицами
\usepackage{array,tabularx,tabulary,booktabs} % Дополнительная работа с таблицами
\usepackage{longtable}  % Длинные таблицы
\usepackage{multirow} % Слияние строк в таблице

\title{ЗАНЯТИЯ СО ЗВЕЗДОЧКОЙ\\СУНЦ 10 КЛАСС}
\date{}

\usepackage{titlesec}
\titleformat*{\section}{\large\bfseries}



\begin{document}
	\maketitle
	
	\section{МАТЕМАТИЧЕСКАЯ СПРАВКА. ЧИСЛА}
	
	Мы уже давно знакомы с этими множествами чисел, но на всякий случай, еще раз вспомним о них:
	
	$\mathbb{N} = \{1, 2, 3, 4, \ldots\}$ - натуральные числа
	
	$\mathbb{Z} = \{\ldots, -2, -1, 0, 1, 2, \ldots\}$ - целые числа
	
	$\mathbb{Q} = \{\frac{m}{n} \quad | \quad m \in \mathbb{Z}, n \in \mathbb{N}\}$ - рациональные числа
	
	$\mathbb{R} = \{a_0,a_1a_2a_3a_4\ldots |  a_0 \in \mathbb{Z}, a_i \in \{0,\ldots,9\}\}$, при этом период из $9$ запрещен - действительные (вещественные) числа ($\mathbb{R}$ от агл. Real)
	
	Заметим, что $\forall a, b \in \mathbb{Q}, a \neq b: \quad \exists c \in \mathbb{Q}: a < c < b$. Именно для того, чтобы это свойство сохранялось, в $\mathbb{R}$ запрещен период из $9$, иначе между $0,(9)$ и $1$ не было бы никакого числа. В множестве рациональных чисел у любого ненулевого элемента есть обратный, то есть $\forall 0 \neq x \in \mathbb{Q} \quad \exists b \in \mathbb{Q}: a \cdot b = b \cdot a = 1$. Но рациональных чисел нам все равно недостаточно для вычислений: например, рациональные числа не содержат в себе длину диагонали квадрата со стороной $1$ ($\sqrt{2} \notin \mathbb{Q}$).
	
	В $\mathbb{R}$ $\forall X, Y: \forall x \in X \ \forall y \in Y \ \exists z: x \leq z \leq y$ (аксиома полноты). Этого свойства, казалось бы, уже достаточно для всего, но:
	
	\begin{displayquote}
	\textit{640 Кб должно быть достаточно для каждого} (Билл Гейтс, 1981)
	\end{displayquote}
	
	
	\subsection{Комплексные числа}
	
	Действительные числа действительно хороши, но у них есть некоторые проблемы. Например, многочлен $x^2 + 1$ не имеет в $\mathbb{R}$ корней. На помощь должно прийти некоторое новое числовое множество.
		
	\begin{definition}\label{4.23p154}
		Множество комплексных чисел $\mathbb{C}$ - это множество с операциями $+$ и $\cdot$, обладающее следующими свойствами:
		
		1) оно содержит в себе множество действительных чисел $\mathbb{R}$ и наследует его свойства сложения и умножения;
		
		2) оно содержит такой элемент $i$, что $i^2 = -1$;
	\end{definition}
	
	Прошу обратить внимание на то, что определение не является формальным, хоть довольно близко к нему.
	
	Комплексные числа можно представлять как множество $\mathbb{C} = \{a+bi | a, b \in \mathbb{R}\}$. В таком случае, результатом сложения комплексного числа $z_1 = a_1 + b_1i$ на $z_2 = a_2 + b_2i$ будет число $z_1 + z_2 = (a_1 + a_2) + (b_1 + b_2)i$, а результатом умножения тех же чисел друг на друга будет являться $z_1 \cdot z_2 = (a_1a_2 - b_1b_2) + (a_1b_2 + b_1a_2)i$ (в этом легко убедиться, раскрыв скобки и воспользовавшись равенством $i^2 = -1$)
    
    
    \begin{definition}\label{4.23p154}
    	$z = a+bi\in \mathbb{C}$.
    	
    	$Re \, z = a$ - вещественная часть числа $z$;
    	
    	$Im \, z = b$ - мнимая часть числа $z$;
    	
    	$\overline{z} = a - bi$ - комплексно сопряженное к $z$ число.
    	
    \end{definition}
    	
   	Заметим, что $\overline{\overline{z}} = z$. Исходя из этого получаем, что $z + \overline{z} \in \mathbb{R}$, а также что \mbox{$z\cdot \overline{z} = a^2+b^2 \in \mathbb{R}$}.
   	
   	Вещественные числа проще всего представлять как точки на прямой, где числу в соответствие ставится точка на прямой с координатой, равной этому числу. Как же геометрически можно представить комплексное число? Заметим, что число $z$ однозначно задается своими вещественной и мнимой частями, так что попробуем рассмотреть плоскость $Oxy$, где оси $x$ и $y$ соответствуют $Re \, z$ и $Im \, z$:
   	
   	\begin{center}
   	% Оси координат:
   	\begin{picture}(80,80)%
   		\put(20,0){\vector(0,1){80}}
   		\put(0,20){\vector(1,0){80}}
   		\put(60,5){$Re \, z$} \put(-7,71){$Im \, z$}
   		\put(20, 20){\vector(1, 2){20}}
   		\put(43,50){$z(a, b)$}
    \end{picture}
    \end{center}

	Оказывается, комплексное число можно спокойно представить в виде вектора с координатами $(Re \, z, Im \, z)$. Как мы знаем, положение точки на плоскости можно задать с помощью чисел $r$ и $\varphi$, где $r$ - длина вектора, а $\varphi$ - угол, составляемый им с горизонтальной осью. В таком случае число $z = a + bi$ представляется в виде
	$$z = r(\cos\varphi + i\sin\varphi),$$
	такое представление комплексного числа называется его тригонометрической формой.
	
	\begin{definition}\label{4.23p154}
		$z = r(\cos\varphi + i\sin\varphi)$.	
		
		$|z| = r$ - модуль числа $z$;
		
		$\arg z = \varphi$ - аргумент числа $z$.
	\end{definition}
	\noindent Легко понять, что $|z| = \sqrt{a^2 + b^2} = \sqrt{z\cdot\overline{z}}$. Чем же нам так удобна тригонометрическая форма записи комплексного числа? Попробуем умножить числа $z_1$ и $z_2$ в тригонометрической форме:
	$$z_1\cdot z_2 = r_1(\cos\varphi_1 + i \sin\varphi_1)r_2(\cos\varphi_2 + i \sin\varphi_2)$$
	Несложными преобразованиями получаем
	\begin{multline*}
	z_1\cdot z_2 = r_1r_2(\cos\varphi_1\cos\varphi_2-\sin\varphi_1\sin\varphi_2 \\
	+ i(\cos\varphi_1\sin\varphi_2 + \cos\varphi_2\sin\varphi_1))
	\end{multline*}
	Вспоминая формулы синуса и косинуса суммы, замечаем, что $$z_1\cdot z_2 = r_1r_2(\cos(\varphi_1 + \varphi_2) + i\sin(\varphi_1 + \varphi_2))$$
	и приходим к тому, что при умножении двух комплексных чисел их модули умножаются, а аргументы складываются.
	
	Отсюда следует $\textit{формула Муавра:}$ $$z^n = r^n(\cos n\varphi + i\sin n\varphi), n \in \mathbb{Z}$$
	
	Теперь рассмотрим следующую задачу: дано число $z \in \mathbb{C}$. Необходимо найти все числа $x$ такие, что $x^n = z$. Иначе говоря, мы хотим найти все корни степени $n$ числа $z$. Сделать это очень просто из формулы Муавра:
	$$x = \sqrt[n]{r}\Big(\cos\big(\frac{\varphi + 2\pi k}{n}\big)+i\sin\big(\frac{\varphi + 2\pi k}{n}\big) \Big),$$
	где $k \in \mathbb{Z}$.
	
	Теперь мы хотим как-то определить комплексную экспоненту $e^{x}$, где $x = a + i\varphi \in \mathbb{C}$. Главное свойство, которое должно выполняться, состоит в том, что $e^{x_1} \cdot e^{x_2} = e^{x_1 + x_2}$. Отсюда должно следовать, что $|e^x| = e^a$ и $e^x = e^a e^{i\varphi} = |e^x|e^{i\varphi} = re^{i\varphi}$. 
	
	Чтобы еще ближе подобраться к определению комплексной экспоненты, еще раз запишем произведение $z_1$ и $z_2$:
	\mbox{$z_1 \cdot z_2 = r_1r_2e^{i(\varphi_1 + \varphi_2)}$}, что должно наводить нас на следующую формулу:
	
	$$e^{i\varphi} = \cos\varphi + i \sin\varphi \ \textit{ (формула Эйлера)}$$
	
	\begin{definition}
		Запись $z = re^{i\varphi}$ называется показательной формой комплексного числа $z$. Так же, как в тригонометрической форме, здесь $r = |z|$, а $\varphi = \arg z$.
	\end{definition}
	
	\subsection{Задачи}
	\begin{enumerate}
		\item Покажите, что геометрически происходит при умножении комплексных чисел с модулем $1$.
		\item Покажите, где в плоскости лежат корни комплексного числа $z$ степени $n$.
		\item Вычислите:
		\begin{enumerate}
			\item $10 * (5 + 3i)$
			\item $i + (10 -2i)$
			\item $i * (11 + i)$
			\item $(10+10i)^3$
		\end{enumerate}
	
		\item Запишите в тригонометрической и показательной формах следующие числа:
		\begin{enumerate}
			\item $1 + \sqrt{3}$
			\item $2i$
			\item $-7i$
			\item $1-\sqrt{3}i$
			\item $3 + 4i$
		\end{enumerate}
	
		\item Найдите частное, домножив знаменатель на комплексно сопряженное:
		\begin{enumerate}
			\item $\frac{1}{i}$
			\item $\frac{1}{1 + i}$
			\item $\frac{1 + i}{2 - i}$
			\item $\frac{a_1 + b_1i}{a_2 + b_2 i}$
		\end{enumerate}
		\item Что происходит при делении чисел в тригонометрической форме?
		\item Найдите все значения корней из комплексных чисел:
		\begin{enumerate}
			\item $\sqrt[2]{1 + \sqrt{3}i}$
			\item $\sqrt[3]{2i}$
			\item $\sqrt[6]{-7i}$
		\end{enumerate}
		\item Пользуясь формулой Эйлера, выразите $\sin x$ и $\cos x$ через комплексные экспоненты.
		\item Решите уравнение: $$x^4 + 2x^2 + 1 = 0$$
	\end{enumerate}
\end{document}